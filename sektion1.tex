% !TEX root = document.tex

\section{Introduktion}

\begin{defn}
Dette er en definition.
\end{defn}

\begin{thm}
Dette er en sætning.
\end{thm}

\begin{lemma}
Dette er et lemma.
\end{lemma}

\begin{defn}
Dette er anden en definition.
\end{defn}

\begin{thm}
Dette er anden en sætning.
\end{thm}

\begin{lemma}
Dette er et andet lemma.
\end{lemma}

\section{Lidt matematik}

\subsection{Kvantorer}
\begin{align*}
\forall x \in X, \quad \exists y \in Y, \quad \text{og} \quad \nexists z \in Z
\end{align*}

\subsection{Nummererede ligninger}
\begin{align}
\int_0^1 x^2 \, dx = \frac{1}{3}
\end{align}

\subsection{Ikke-nummererede ligninger}
\begin{align*}
f(x) &= x^2 \\
g(x) &= \sin(x) \\
h(x) &= \log(x)
\end{align*}


\subsection{Mængdenotation}
\begin{align*}
A &= \{1, 2, 3\} \\
B &= \{x \in \mathbb{R} \mid x > 0\} \\
C &= \emptyset
A \cup B &= \{1, 2, 3, x \in \mathbb{R} \mid x > 0\}
\end{align*}


\subsection{Funktioner}
\begin{align*}
f(x) &= x^2 \\
f'(x) &= 2x \\
f''(x) &= 2
\end{align*}


\subsection{Integralregning}
\begin{align*}
\int_0^1 x^2 \, dx = \frac{1}{3}
\end{align*}


\subsection{Matricer}
\begin{align*}
A &= \begin{bmatrix}
1 & 2 \\
3 & 4
\end{bmatrix} \\
B &= \begin{bmatrix}
5 & 6 \\
7 & 8
\end{bmatrix} \\
C &= A + B = \begin{bmatrix}
6 & 8 \\
10 & 12
\end{bmatrix}
\end{align*}

